%% Generated by Sphinx.
\def\sphinxdocclass{report}
\documentclass[letterpaper,10pt,english,openany,oneside]{sphinxmanual}
\ifdefined\pdfpxdimen
   \let\sphinxpxdimen\pdfpxdimen\else\newdimen\sphinxpxdimen
\fi \sphinxpxdimen=.75bp\relax

\PassOptionsToPackage{warn}{textcomp}
\usepackage[utf8]{inputenc}
\ifdefined\DeclareUnicodeCharacter
% support both utf8 and utf8x syntaxes
  \ifdefined\DeclareUnicodeCharacterAsOptional
    \def\sphinxDUC#1{\DeclareUnicodeCharacter{"#1}}
  \else
    \let\sphinxDUC\DeclareUnicodeCharacter
  \fi
  \sphinxDUC{00A0}{\nobreakspace}
  \sphinxDUC{2500}{\sphinxunichar{2500}}
  \sphinxDUC{2502}{\sphinxunichar{2502}}
  \sphinxDUC{2514}{\sphinxunichar{2514}}
  \sphinxDUC{251C}{\sphinxunichar{251C}}
  \sphinxDUC{2572}{\textbackslash}
\fi
\usepackage{cmap}
\usepackage[T1]{fontenc}
\usepackage{amsmath,amssymb,amstext}
\usepackage{babel}



\usepackage{times}
\expandafter\ifx\csname T@LGR\endcsname\relax
\else
% LGR was declared as font encoding
  \substitutefont{LGR}{\rmdefault}{cmr}
  \substitutefont{LGR}{\sfdefault}{cmss}
  \substitutefont{LGR}{\ttdefault}{cmtt}
\fi
\expandafter\ifx\csname T@X2\endcsname\relax
  \expandafter\ifx\csname T@T2A\endcsname\relax
  \else
  % T2A was declared as font encoding
    \substitutefont{T2A}{\rmdefault}{cmr}
    \substitutefont{T2A}{\sfdefault}{cmss}
    \substitutefont{T2A}{\ttdefault}{cmtt}
  \fi
\else
% X2 was declared as font encoding
  \substitutefont{X2}{\rmdefault}{cmr}
  \substitutefont{X2}{\sfdefault}{cmss}
  \substitutefont{X2}{\ttdefault}{cmtt}
\fi


\usepackage[Bjarne]{fncychap}
\usepackage{sphinx}

\fvset{fontsize=\small}
\usepackage{geometry}


% Include hyperref last.
\usepackage{hyperref}
% Fix anchor placement for figures with captions.
\usepackage{hypcap}% it must be loaded after hyperref.
% Set up styles of URL: it should be placed after hyperref.
\urlstyle{same}

\usepackage{sphinxmessages}
\setcounter{tocdepth}{3}
\setcounter{secnumdepth}{3}


\title{VHH Plugin Package: Shot Type Classification (vhh\_stc)}
\date{Jun 05, 2020}
\release{1.0.0}
\author{Daniel Helm}
\newcommand{\sphinxlogo}{\vbox{}}
\renewcommand{\releasename}{Release}
\makeindex
\begin{document}

\pagestyle{empty}
\sphinxmaketitle
\pagestyle{plain}
\sphinxtableofcontents
\pagestyle{normal}
\phantomsection\label{\detokenize{index::doc}}


The following list give an overview of the folder structure of this python repository:

\sphinxstyleemphasis{name of repository}: vhh\_stc
\begin{itemize}
\item {} 
\sphinxstylestrong{ApiSphinxDocumentation/}: includes all files to generate the documentation as well as the created documentations (html, pdf)

\item {} 
\sphinxstylestrong{config/}: this folder includes the required configuration file

\item {} 
\sphinxstylestrong{stc/}: this folder represents the shot\sphinxhyphen{}type\sphinxhyphen{}classification module and builds the main part of this repository

\item {} 
\sphinxstylestrong{Demo/}: this folder includes a demo script to demonstrate how the package have to be used in customized applications

\item {} 
\sphinxstylestrong{Develop/}: includes scripts to train and evaluate the pytorch models. Furthermore, a script is included to create the package documentation (pdf, html)

\item {} 
\sphinxstylestrong{README.md}: this file gives a brief description of this repository (e.g. link to this documentation)

\item {} 
\sphinxstylestrong{requirements.txt}: this file holds all python lib dependencies and is needed to install the package in your own virtual environment

\item {} 
\sphinxstylestrong{setup.py}: this script is needed to install the stc package in your own virtual environment

\end{itemize}


\chapter{Setup  instructions}
\label{\detokenize{index:setup-instructions}}
This package includes a setup.py script and a requirements.txt file which are needed to install this package for custom applications.
The following instructions have to be done to used this library in your own application:

Requirements:
\begin{itemize}
\item {} 
Ubuntu 18.04 LTS

\item {} 
CUDA 10.1 + cuDNN

\item {} 
python version 3.6.x

\end{itemize}

Create a virtual environment:
\begin{itemize}
\item {} 
create a folder to a specified path (e.g. /xxx/vhh\_stc/)

\item {} 
python3 \sphinxhyphen{}m venv /xxx/vhh\_stc/

\end{itemize}

Activate the environment:
\begin{itemize}
\item {} 
source /xxx/vhh\_stc/bin/activate

\end{itemize}

Checkout vhh\_stc repository to a specified folder:
\begin{itemize}
\item {} 
git clone \sphinxurl{https://github.com/dahe-cvl/vhh\_stc}

\end{itemize}

Install the stc package and all dependencies:
\begin{itemize}
\item {} 
change to the root directory of the repository (includes setup.py)

\item {} 
python setup.py install

\end{itemize}

\begin{sphinxadmonition}{note}{Note:}
You can check the success of the installation by using the commend \sphinxstyleemphasis{pip list}. This command should give you a list with all installed python packages and it should include \sphinxstyleemphasis{vhh\_stc}
\end{sphinxadmonition}

\begin{sphinxadmonition}{note}{Note:}
Currently there is an issue in the \sphinxstyleemphasis{setup.py} script. Therefore the pytorch libraries have to be installed manually by running the following command:
\sphinxstyleemphasis{pip install torch==1.5.0+cu101 torchvision==0.6.0+cu101 \sphinxhyphen{}f https://download.pytorch.org/whl/torch\_stable.html}
\end{sphinxadmonition}


\chapter{Dataset Generator}
\label{\detokenize{index:dataset-generator}}
In the \sphinxstyleemphasis{Develop/dataset\_annotation\_scripts} helper scripts are included to generate a annotated dataset to train
a the classification model.

\sphinxstylestrong{annotationToolShotTypes\_v2.py}
\begin{quote}

This script provides a simple frame player GUI to iterate over the frames included in a specified folder. Moreover,
each frame can be annotated with a simple keyboard command to configured class names. The keyboard commands are
explained in the script and a configuration section is placed at the beginning of the script. This tool can also be
used in Windows by executing the batch script (python 3.6.x with opencv is required).
\end{quote}

\sphinxstylestrong{extractAnnotatedFrames.py}
\begin{quote}

After the annotation process is finished (result: xxx.csv file including frame ID and class\_name) this script can be
used to extract all annotated frames.
\end{quote}

\sphinxstylestrong{showAnnotatedFrames.py}
\begin{quote}

This script is used to step through all annotated frames.
\end{quote}


\chapter{API Description}
\label{\detokenize{index:api-description}}
This section gives an overview of all classes and modules in \sphinxstyleemphasis{stc} as well as an code description.


\section{Configuration class}
\label{\detokenize{Configuration:configuration-class}}\label{\detokenize{Configuration::doc}}\index{Configuration (class in stc.Configuration)@\spxentry{Configuration}\spxextra{class in stc.Configuration}}

\begin{fulllineitems}
\phantomsection\label{\detokenize{Configuration:stc.Configuration.Configuration}}\pysiglinewithargsret{\sphinxbfcode{\sphinxupquote{class }}\sphinxcode{\sphinxupquote{stc.Configuration.}}\sphinxbfcode{\sphinxupquote{Configuration}}}{\emph{\DUrole{n}{config\_file}\DUrole{p}{:} \DUrole{n}{str}}}{}
Bases: \sphinxcode{\sphinxupquote{object}}

This class is needed to read the configuration parameters specified in the configuration.yaml file.
The instance of the class is holding all parameters during runtime.

\begin{sphinxadmonition}{note}{Note:}
e.g. ./config/config\_vhh\_test.yaml
\begin{quote}

the yaml file is separated in multiple sections
config{[}‘Development’{]}
config{[}‘PreProcessing’{]}
config{[}‘StcCore’{]}
config{[}‘Evaluation’{]}

whereas each section should hold related and meaningful parameters.
\end{quote}
\end{sphinxadmonition}
\index{loadConfig() (stc.Configuration.Configuration method)@\spxentry{loadConfig()}\spxextra{stc.Configuration.Configuration method}}

\begin{fulllineitems}
\phantomsection\label{\detokenize{Configuration:stc.Configuration.Configuration.loadConfig}}\pysiglinewithargsret{\sphinxbfcode{\sphinxupquote{loadConfig}}}{}{}
Method to load configurables from the specified configuration file

\end{fulllineitems}


\end{fulllineitems}



\section{STC class}
\label{\detokenize{STC:stc-class}}\label{\detokenize{STC::doc}}\index{STC (class in stc.STC)@\spxentry{STC}\spxextra{class in stc.STC}}

\begin{fulllineitems}
\phantomsection\label{\detokenize{STC:stc.STC.STC}}\pysiglinewithargsret{\sphinxbfcode{\sphinxupquote{class }}\sphinxcode{\sphinxupquote{stc.STC.}}\sphinxbfcode{\sphinxupquote{STC}}}{\emph{\DUrole{n}{config\_file}\DUrole{p}{:} \DUrole{n}{str}}}{}
Bases: \sphinxcode{\sphinxupquote{object}}

Main class of shot type classification (stc) package.
\index{exportStcResults() (stc.STC.STC method)@\spxentry{exportStcResults()}\spxextra{stc.STC.STC method}}

\begin{fulllineitems}
\phantomsection\label{\detokenize{STC:stc.STC.STC.exportStcResults}}\pysiglinewithargsret{\sphinxbfcode{\sphinxupquote{exportStcResults}}}{\emph{\DUrole{n}{fName}}, \emph{\DUrole{n}{stc\_results\_np}\DUrole{p}{:} \DUrole{n}{numpy.ndarray}}}{}
Method to export stc results as csv file.
\begin{quote}\begin{description}
\item[{Parameters}] \leavevmode\begin{itemize}
\item {} 
\sphinxstyleliteralstrong{\sphinxupquote{fName}} \textendash{} {[}required{]} name of result file.

\item {} 
\sphinxstyleliteralstrong{\sphinxupquote{stc\_results\_np}} \textendash{} numpy array holding the shot type classification predictions for each shot of a movie.

\end{itemize}

\end{description}\end{quote}

\end{fulllineitems}

\index{loadSbdResults() (stc.STC.STC method)@\spxentry{loadSbdResults()}\spxextra{stc.STC.STC method}}

\begin{fulllineitems}
\phantomsection\label{\detokenize{STC:stc.STC.STC.loadSbdResults}}\pysiglinewithargsret{\sphinxbfcode{\sphinxupquote{loadSbdResults}}}{\emph{\DUrole{n}{sbd\_results\_path}}}{}
Method for loading shot boundary detection results as numpy array

\begin{sphinxadmonition}{note}{Note:}
Only used in debug\_mode.
\end{sphinxadmonition}
\begin{quote}\begin{description}
\item[{Parameters}] \leavevmode
\sphinxstyleliteralstrong{\sphinxupquote{sbd\_results\_path}} \textendash{} {[}required{]} path to results file of shot boundary detection module (vhh\_sbd)

\item[{Returns}] \leavevmode
numpy array holding list of detected shots.

\end{description}\end{quote}

\end{fulllineitems}

\index{runModel() (stc.STC.STC method)@\spxentry{runModel()}\spxextra{stc.STC.STC method}}

\begin{fulllineitems}
\phantomsection\label{\detokenize{STC:stc.STC.STC.runModel}}\pysiglinewithargsret{\sphinxbfcode{\sphinxupquote{runModel}}}{\emph{\DUrole{n}{model}}, \emph{\DUrole{n}{tensor\_l}}}{}
Method to calculate stc predictions of specified model and given list of tensor images (pytorch).
\begin{quote}\begin{description}
\item[{Parameters}] \leavevmode\begin{itemize}
\item {} 
\sphinxstyleliteralstrong{\sphinxupquote{model}} \textendash{} {[}required{]} pytorch model instance

\item {} 
\sphinxstyleliteralstrong{\sphinxupquote{tensor\_l}} \textendash{} {[}required{]} list of tensors representing a list of frames.

\end{itemize}

\item[{Returns}] \leavevmode
predicted class\_name for each tensor frame,
the number of hits within a shot,
frame\sphinxhyphen{}based predictions for a whole shot

\end{description}\end{quote}

\end{fulllineitems}

\index{runOnSingleVideo() (stc.STC.STC method)@\spxentry{runOnSingleVideo()}\spxextra{stc.STC.STC method}}

\begin{fulllineitems}
\phantomsection\label{\detokenize{STC:stc.STC.STC.runOnSingleVideo}}\pysiglinewithargsret{\sphinxbfcode{\sphinxupquote{runOnSingleVideo}}}{\emph{\DUrole{n}{shots\_per\_vid\_np}\DUrole{o}{=}\DUrole{default_value}{None}}, \emph{\DUrole{n}{max\_recall\_id}\DUrole{o}{=}\DUrole{default_value}{\sphinxhyphen{} 1}}}{}
Method to run stc classification on specified video.
\begin{quote}\begin{description}
\item[{Parameters}] \leavevmode\begin{itemize}
\item {} 
\sphinxstyleliteralstrong{\sphinxupquote{shots\_per\_vid\_np}} \textendash{} {[}required{]} numpy array representing all detected shots in a video
(e.g. sid | movie\_name | start | end )

\item {} 
\sphinxstyleliteralstrong{\sphinxupquote{max\_recall\_id}} \textendash{} {[}required{]} integer value holding unique video id from VHH MMSI system

\end{itemize}

\end{description}\end{quote}

\end{fulllineitems}


\end{fulllineitems}



\section{Video class}
\label{\detokenize{Video:video-class}}\label{\detokenize{Video::doc}}\index{Video (class in stc.Video)@\spxentry{Video}\spxextra{class in stc.Video}}

\begin{fulllineitems}
\phantomsection\label{\detokenize{Video:stc.Video.Video}}\pysigline{\sphinxbfcode{\sphinxupquote{class }}\sphinxcode{\sphinxupquote{stc.Video.}}\sphinxbfcode{\sphinxupquote{Video}}}
Bases: \sphinxcode{\sphinxupquote{object}}

This class is representing a video. Each instance of this class is holding the properties of one Video.
\index{getFrame() (stc.Video.Video method)@\spxentry{getFrame()}\spxextra{stc.Video.Video method}}

\begin{fulllineitems}
\phantomsection\label{\detokenize{Video:stc.Video.Video.getFrame}}\pysiglinewithargsret{\sphinxbfcode{\sphinxupquote{getFrame}}}{\emph{\DUrole{n}{frame\_id}}}{}
Method to get one frame of a video on a specified position.
\begin{quote}\begin{description}
\item[{Parameters}] \leavevmode
\sphinxstyleliteralstrong{\sphinxupquote{frame\_id}} \textendash{} {[}required{]} integer value with valid frame index

\item[{Returns}] \leavevmode
numpy frame (WxHx3)

\end{description}\end{quote}

\end{fulllineitems}

\index{load() (stc.Video.Video method)@\spxentry{load()}\spxextra{stc.Video.Video method}}

\begin{fulllineitems}
\phantomsection\label{\detokenize{Video:stc.Video.Video.load}}\pysiglinewithargsret{\sphinxbfcode{\sphinxupquote{load}}}{\emph{\DUrole{n}{vidFile}\DUrole{p}{:} \DUrole{n}{str}}}{}
Method to load video file.
\begin{quote}\begin{description}
\item[{Parameters}] \leavevmode
\sphinxstyleliteralstrong{\sphinxupquote{vidFile}} \textendash{} {[}required{]} string representing path to video file

\end{description}\end{quote}

\end{fulllineitems}

\index{printVIDInfo() (stc.Video.Video method)@\spxentry{printVIDInfo()}\spxextra{stc.Video.Video method}}

\begin{fulllineitems}
\phantomsection\label{\detokenize{Video:stc.Video.Video.printVIDInfo}}\pysiglinewithargsret{\sphinxbfcode{\sphinxupquote{printVIDInfo}}}{}{}
Method to a print summary of video properties.

\end{fulllineitems}


\end{fulllineitems}



\section{Models \sphinxhyphen{} module}
\label{\detokenize{Models:module-stc.Models}}\label{\detokenize{Models:models-module}}\label{\detokenize{Models::doc}}\index{module@\spxentry{module}!stc.Models@\spxentry{stc.Models}}\index{stc.Models@\spxentry{stc.Models}!module@\spxentry{module}}\index{loadModel() (in module stc.Models)@\spxentry{loadModel()}\spxextra{in module stc.Models}}

\begin{fulllineitems}
\phantomsection\label{\detokenize{Models:stc.Models.loadModel}}\pysiglinewithargsret{\sphinxcode{\sphinxupquote{stc.Models.}}\sphinxbfcode{\sphinxupquote{loadModel}}}{\emph{\DUrole{n}{model\_arch}\DUrole{o}{=}\DUrole{default_value}{\textquotesingle{}\textquotesingle{}}}, \emph{\DUrole{n}{classes}\DUrole{o}{=}\DUrole{default_value}{None}}, \emph{\DUrole{n}{pre\_trained\_path}\DUrole{o}{=}\DUrole{default_value}{None}}}{}
This module is used to load specified deep learning model.
\begin{quote}\begin{description}
\item[{Parameters}] \leavevmode\begin{itemize}
\item {} 
\sphinxstyleliteralstrong{\sphinxupquote{model\_arch}} \textendash{} string value {[}required{]} \sphinxhyphen{} is used to select between various deep learning architectures
(Resnet, Vgg, Densenet, Alexnet)

\item {} 
\sphinxstyleliteralstrong{\sphinxupquote{classes}} \textendash{} list of strings {[}required{]} \sphinxhyphen{} is used to hold the class names (e.g. {[}‘ELS’, ‘LS’, ‘MS’, ‘CU’{]})

\item {} 
\sphinxstyleliteralstrong{\sphinxupquote{pre\_trained\_path}} \textendash{} string {[}optional{]} \sphinxhyphen{} is used to specify the path to a pre\sphinxhyphen{}trained model

\end{itemize}

\item[{Returns}] \leavevmode
the specified instance of the model

\end{description}\end{quote}

\end{fulllineitems}



\section{Datasets module}
\label{\detokenize{Datasets:module-stc.Datasets}}\label{\detokenize{Datasets:datasets-module}}\label{\detokenize{Datasets::doc}}\index{module@\spxentry{module}!stc.Datasets@\spxentry{stc.Datasets}}\index{stc.Datasets@\spxentry{stc.Datasets}!module@\spxentry{module}}\index{loadDatasetFromFolder() (in module stc.Datasets)@\spxentry{loadDatasetFromFolder()}\spxextra{in module stc.Datasets}}

\begin{fulllineitems}
\phantomsection\label{\detokenize{Datasets:stc.Datasets.loadDatasetFromFolder}}\pysiglinewithargsret{\sphinxcode{\sphinxupquote{stc.Datasets.}}\sphinxbfcode{\sphinxupquote{loadDatasetFromFolder}}}{\emph{\DUrole{n}{path}\DUrole{o}{=}\DUrole{default_value}{\textquotesingle{}\textquotesingle{}}}, \emph{\DUrole{n}{batch\_size}\DUrole{o}{=}\DUrole{default_value}{64}}}{}
This method is used to load a specified dataset.
\begin{quote}\begin{description}
\item[{Parameters}] \leavevmode\begin{itemize}
\item {} 
\sphinxstyleliteralstrong{\sphinxupquote{path}} \textendash{} {[}required{]} path to dataset folder holding the subfolders “train”, “val” and “test”.

\item {} 
\sphinxstyleliteralstrong{\sphinxupquote{batch\_size}} \textendash{} {[}optional{]} specifies the batchsize used during training process.

\end{itemize}

\item[{Returns}] \leavevmode
instance of trainloader, validloader, testloader as well as the corresponding dataset sizes

\end{description}\end{quote}

\end{fulllineitems}



\section{CustomTransforms class}
\label{\detokenize{CustomTransforms:customtransforms-class}}\label{\detokenize{CustomTransforms::doc}}\index{ToGrayScale (class in stc.CustomTransforms)@\spxentry{ToGrayScale}\spxextra{class in stc.CustomTransforms}}

\begin{fulllineitems}
\phantomsection\label{\detokenize{CustomTransforms:stc.CustomTransforms.ToGrayScale}}\pysigline{\sphinxbfcode{\sphinxupquote{class }}\sphinxcode{\sphinxupquote{stc.CustomTransforms.}}\sphinxbfcode{\sphinxupquote{ToGrayScale}}}
Bases: \sphinxcode{\sphinxupquote{object}}

This class is needed to transform rbg numpy frames to grayscale numpys during the training process with pytorch.

\end{fulllineitems}



\section{Shot class}
\label{\detokenize{Shot:shot-class}}\label{\detokenize{Shot::doc}}\index{Shot (class in stc.Shot)@\spxentry{Shot}\spxextra{class in stc.Shot}}

\begin{fulllineitems}
\phantomsection\label{\detokenize{Shot:stc.Shot.Shot}}\pysiglinewithargsret{\sphinxbfcode{\sphinxupquote{class }}\sphinxcode{\sphinxupquote{stc.Shot.}}\sphinxbfcode{\sphinxupquote{Shot}}}{\emph{\DUrole{n}{sid}}, \emph{\DUrole{n}{movie\_name}}, \emph{\DUrole{n}{start\_pos}}, \emph{\DUrole{n}{end\_pos}}}{}
Bases: \sphinxcode{\sphinxupquote{object}}

This class is representing a shot. Each instance of this class is holding the properties of one shot.
\index{convert2String() (stc.Shot.Shot method)@\spxentry{convert2String()}\spxextra{stc.Shot.Shot method}}

\begin{fulllineitems}
\phantomsection\label{\detokenize{Shot:stc.Shot.Shot.convert2String}}\pysiglinewithargsret{\sphinxbfcode{\sphinxupquote{convert2String}}}{}{}
Method to convert class member properties in a semicolon separated string.
\begin{quote}\begin{description}
\item[{Returns}] \leavevmode
string holding all properties of one shot.

\end{description}\end{quote}

\end{fulllineitems}

\index{printShotInfo() (stc.Shot.Shot method)@\spxentry{printShotInfo()}\spxextra{stc.Shot.Shot method}}

\begin{fulllineitems}
\phantomsection\label{\detokenize{Shot:stc.Shot.Shot.printShotInfo}}\pysiglinewithargsret{\sphinxbfcode{\sphinxupquote{printShotInfo}}}{}{}
Method to a print summary of shot properties.

\end{fulllineitems}


\end{fulllineitems}



\chapter{Indices and tables}
\label{\detokenize{index:indices-and-tables}}\begin{itemize}
\item {} 
\DUrole{xref,std,std-ref}{genindex}

\item {} 
\DUrole{xref,std,std-ref}{modindex}

\item {} 
\DUrole{xref,std,std-ref}{search}

\end{itemize}


\section{References}
\label{\detokenize{index:references}}

\renewcommand{\indexname}{Python Module Index}
\begin{sphinxtheindex}
\let\bigletter\sphinxstyleindexlettergroup
\bigletter{s}
\item\relax\sphinxstyleindexentry{stc.Datasets}\sphinxstyleindexpageref{Datasets:\detokenize{module-stc.Datasets}}
\item\relax\sphinxstyleindexentry{stc.Models}\sphinxstyleindexpageref{Models:\detokenize{module-stc.Models}}
\end{sphinxtheindex}

\renewcommand{\indexname}{Index}
\printindex
\end{document}